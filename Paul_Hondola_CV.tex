%-------------------------
% Author : Paul Hondola
% License : MIT
%------------------------

\documentclass[letterpaper, 11pt]{article}

\usepackage{latexsym}
\usepackage[empty]{fullpage}
\usepackage{titlesec}
\usepackage{marvosym}
\usepackage[usenames, dvipsnames]{color}
\usepackage{verbatim}
\usepackage{enumitem}
\usepackage[hidelinks]{hyperref}
\usepackage{fancyhdr}
\usepackage[english]{babel}
\usepackage{tabularx}
\input{glyphtounicode}

%----------FONT OPTIONS----------
% sans-serif
% \usepackage[sfdefault]{FiraSans}
% \usepackage[sfdefault]{roboto}
% \usepackage[sfdefault]{noto-sans}
% \usepackage[default]{sourcesanspro}

% serif
% \usepackage{CormorantGaramond}
% \usepackage{charter}

\pagestyle{fancy}
\fancyhf{}
\fancyfoot{}
\renewcommand{\headrulewidth}{0pt}
\renewcommand{\footrulewidth}{0pt}

% Adjust margins
\addtolength{\oddsidemargin}{-0.5in}
\addtolength{\evensidemargin}{-0.5in}
\addtolength{\textwidth}{1in}
\addtolength{\topmargin}{-.5in}
\addtolength{\textheight}{1.0in}

\urlstyle{same}

\raggedbottom
\raggedright
\setlength{\tabcolsep}{0in}

% Sections formatting
\titleformat{\section}{ \vspace{-4pt}\scshape\raggedright\large }{}{0em}{}[
\color{black}
\titlerule
\vspace{-5pt}
]

% Ensure that generate pdf is machine readable/ATS parsable
\pdfgentounicode=1

%-------------------------
% Custom commands
\newcommand{\resumeItem}[1]{ \item\small{ {#1 \vspace{-2pt}} } }

\newcommand{\resumeSubheading}[4]{
\vspace{-2pt}
\item
\begin{tabular*}{0.97\textwidth}[t]{l@{\extracolsep{\fill}}r}
	\textbf{#1}       & #2                 \\
	\textit{\small#3} & \textit{\small #4} \\
\end{tabular*}
\vspace{-7pt}
}

\newcommand{\resumeSubSubheading}[2]{ \item
\begin{tabular*}{0.97\textwidth}{l@{\extracolsep{\fill}}r}
	\textit{\small#1} & \textit{\small #2} \\
\end{tabular*}
\vspace{-7pt}
}

\newcommand{\resumeProjectHeading}[2]{ \item
\begin{tabular*}{0.97\textwidth}{l@{\extracolsep{\fill}}r}
	\small#1 & #2 \\
\end{tabular*}
\vspace{-7pt}
}

\newcommand{\resumeSubItem}[1]{\resumeItem{#1}
\vspace{-4pt}}

\renewcommand{\labelitemii}{$\vcenter{\hbox{\tiny$\bullet$}}$}

\newcommand{\resumeSubHeadingListStart}{\begin{itemize}[leftmargin=0.15in, label={}]}
\newcommand{\resumeSubHeadingListEnd}{\end{itemize}}
\newcommand{\resumeItemListStart}{\begin{itemize}}
\newcommand{\resumeItemListEnd}{\end{itemize}
\vspace{-5pt}}

\begin{document}
	%----------HEADING----------

	\begin{center}
		\textbf{\Huge \scshape Paul Hondola} \\
		\vspace{1pt}
		\small +40745166873 $|$ \href{mailto:paulhondola@gmail.com}{\underline{paulhondola@gmail.com}}
		$|$ \href{https://linkedin.com/in/paulhondola}{\underline{linkedin.com/in/paulhondola}} $|$
		\href{https://github.com/paulhondola}{\underline{github.com/paulhondola}}
	\end{center}

	%-----------EDUCATION-----------
	\section{Education}
	\resumeSubHeadingListStart

	\resumeSubheading{Bachelor of Science in Computer Engineering}{Oct. 2023 -- June 2027}{Universitatea Politehnica Timisoara}{}

	\resumeSubheading{Deepblue Maker - Underwater Robotics Summer Camp}{Jul. 2025 -- Aug. 2025}{Hangzhou Dianzi University, China}{}

	\resumeItemListStart
		\resumeItem{Worked alongside a team of 6 students to design and build an underwater robot that can navigate through water and perform tasks such as mapping the environment and detecting sea life.}
		\resumeItemListStart 
			\resumeItem{Mechanical: CAD design in Solidworks and 3D printing of parts. Physical testing for water sealing and buoyancy.}
			\resumeItem{Electrical: STM32 microcontroller to control the robot's thrusters from a desktop computer.}
			\resumeItem{Software: Raspberry Pi 5 to get a live feed from the robot's camera and display it on a screen. Developed an object detection model using YOLO to detect sea life.}
		\resumeItemListEnd
	\resumeItemListEnd
    
	\resumeSubHeadingListEnd

	%-----------EXPERIENCE-----------
	\section{Experience}
	\resumeSubHeadingListStart

	\resumeSubheading {Full Stack Developer}{Jul. 2025 -- current} {Hibyte $|$ \emph{Typescript, Angular, NestJS, Payload CMS, Supabase, Docker, Github Actions}}{}
    
    \resumeItemListStart 
        \resumeItem{GameBox - A full-stack monorepo for managing a game center, featuring a content management system with admin interface and user-facing web application.}
        
        \resumeItem{Developed with the help of a modern Tech Stack:}
        \resumeItemListStart 
            \resumeItem{Frontend: Angular v20, Typescript, HTML, SCSS}
            \resumeItem{Backend \& Database: NestJS \& Payload CMS \& Supabase}
            \resumeItem{CI/CD: Docker, Github Actions}
        \resumeItemListEnd
    \resumeItemListEnd

	\resumeSubheading {Malware Analyst Trainee}{Apr. 2025 -- Jun. 2025} {Bitdefender $|$ \emph{Java + jadx, C \& x86 Assembly + IDA, Python}}{}

    \resumeItemListStart

    \resumeItem{Participated in Bitdefender's Academic Labs program, focused on reverse engineering and malware analysis. Working hands-on with Windows and Android environments to analyze vulnerabilities, study malware behavior, and explore exploitation techniques.}

    \resumeItemListStart 
                    
            \resumeItem{Studied Android system architecture, its security model and APKs}
    
	        \resumeItem{Developed skills in static and dynamic analysis, decompilation and disassembly}

	          \resumeItem{Reverse engineered simple encryption algorithms inside ransomware}

	\resumeItemListEnd 
    
    \resumeItemListEnd
    
    \resumeSubHeadingListEnd

    %-----------TECHNICAL SKILLS-----------
	\section{Technical Skills}
	\begin{itemize}[leftmargin=0.15in, label={}]
		\small{\item{

		\textbf{Programming Languages}{: C, C++, Python, Java, TypeScript, Bash, SQL} \\
        
        \textbf{Frontend Development}{: Angular, HTML5, SCSS} \\
        
        \textbf{Backend \& Databases}{: NestJS, Supabase, Payload CMS, REST APIs} \\

		\textbf{DevOps \& Tooling}{: Linux, Docker, Git, GitHub Actions, Make, Clang} \\

		\textbf{Languages}{: English C1 (Cambridge Assessment)} \\ }}
	\end{itemize}

	%-----------PROJECTS-----------
	\section{Projects}
	\resumeSubHeadingListStart

	\resumeProjectHeading
	{\textbf{Benchmark Suite} $|$ \emph{Python, psutil, pycpu-info, ML libraries, NumPy, Pandas, Docker}}{May 2025 - Jun 2025}

	\resumeItemListStart

	   \resumeItem{Cross-platform system performance benchmarking suite built for in-depth analysis and comparison of CPU, GPU, memory, and cache performance across workloads, architectures, and environments.}

	   \resumeItem{ Configurable microbenchmarks: floating point throughput, memory latency/bandwidth, thread scalability}

	   \resumeItem{ ML workloads powered by scikit-learn, PyTorch (CPU/GPU/MPS), and TensorFlow}
    
	   \resumeItem{Compiler benchmarking with gcc / clang via real-world C project compilation}

	   \resumeItem{Detailed hardware info introspection (RAM, CPU cores, frequencies, per-core usage, cache levels)}

	\resumeItemListEnd

	\resumeProjectHeading {\textbf{Treasure Hunt System} $|$ \emph{C, POSIX system calls, Clang, Make, Git}}{Mar 2025 -- May 2025}

	\resumeItemListStart 
        
        \resumeItem{Introduces an interactive shell-like CLI program to manage hunts and treasures via commands}

	   \resumeItem{Uses logs to track user operations, with symlinked logs for centralized access}

	   \resumeItem{Utilizes multi-process architecture and sigaction-based signal handling for inter-process communication}

	   \resumeItem{Enables runtime features such as live monitoring, hunt and treasure inspection, and controlled shutdown of the monitor process}

	\resumeItemListEnd

	\resumeProjectHeading
	{\textbf{SafetyMap - Community-driven Safety App} $|$ \emph{Java, Android, Google Maps API, Firebase}}{Nov 2024 -- Nov 2024}

	\resumeItemListStart 
    
        \resumeItem{Interactive Map: Mark and view safety alerts using Google Maps}

	   \resumeItem{User Alerts: Users can drop pins on the map to report issues such as thefts, road hazards, or other dangers}

	   \resumeItem{Notifications: Real-time notifications for users approaching an area with a safety alert}

	   \resumeItem{Community Trust System: Users can vote on the validity of alerts, contributing to a community trust score}

	\resumeItemListEnd

	\resumeProjectHeading {\textbf{FPGA Video Transmission and Image Processing} $|$ \emph{Verilog, VHDL, Xilinx FPGA}}{Mar 2024 -- Jun 2024}

	\resumeItemListStart

	   \resumeItem{Hardware based video transmission and image processing system, with camera input and display via VGA. }

	   \resumeItem{Supports basic image processing and integrates with OpenCV for face recognition through UART.}

	\resumeItemListEnd 
    
    \resumeSubHeadingListEnd

	%-----------CERTIFICATES-----------
	%----------------------------------

	%-------------------------------------------
\end{document}